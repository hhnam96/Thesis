\documentclass[11pt]{article}

\begin{document}

\section{Introduction}

\begin{enumerate}

\item The discovery of Gas Giant Planet in recent year lay the framework and constraint for the analytical and numerical study. 

The field of exoplanet research has been expanding rapidly in the recent decade, enabled by the progress of observation in terms of sheer quantity and quality. Transiting and radial velocity emerge as the main workhorse, having been used in tandem to calculate the orbital and structural features of these exoplanets.  Complementarily, Transmission and emission spectroscopy provide the useful constraint of the temperature profile and atmospheric composition (Heng \& Showman 2015)  Even the observables related to the dynamics like the brightness distribution (Knutson et al. 2007) and preliminary insight about wind speed by the blueshift of CO line (Snell et al. 2010) have also been found.

Among the exoplanets, Hot Jupiter is of particularly interest because of their distinct characteristics. They are in close proximity with their star (within \~ 0.1 AU), enabling observation though transit much easier by geometry. Also the second eclipse depth is also more visble by the hot atmosphere heated. Hot Jupiter therefore remain at the forefront of the study of extrasolar atmospheric circulation.

The intensive stellar irradiation also renders the atmosphere of Hot Jupiter dynamic and vibrant.  It forms a distinct radiative layer atop convective interiors and drives atmospheric wind to unusual high speed. On top of that, because of being tidal locked,which is often the case, their rotation period is equal to their orbital period, leading to a distinct and permanet day side and night side. Such asymmetric heating drives strong wind for heat redistrubution. 

Overall, the thermal and and dynamical responses to the unusually high thermal forcing 

\begin{itemize}
\item Brief characteristics of hot jupiter
\begin{itemize}
\item Tidal lock, so its rotational period is equal to its orbital period. Leading to a distinct day and night side of a planet.
\item Close to the central star => high stellar flux => hot, strong radiative forcing 
\item Roughly same mass with jupiter
\end{itemize}
\item With what mission?
\item With what technique?
\item What have been observed?
\end{itemize}

\item Numerical simulation of such planets have been done by several group with improvement is ongoing implemented.

These appealing features plus the auspicious future in observation that new instruments such as the soon be available James Webb Space Telescope will provide important constraints, make theoretical study of Hot Jupiter a truly hot topic. 

Thus far, atmospheric circulation of Hot Jupiter has been modeled by several groups with ongoing development (Showman et al. 2008, ...). Some robust and consistent results were also prominent in these Global Circulation Models such as the transonic wind speed and most notably is the broad eastward equatorial zonal jet near the photosphere. This zonal wind often flow with the velocity of order few kilometer per second, which is larger than speed of sound by a factor of a few. With this zonal wind, the brightest spot are predicted to be advected eastward of the substellar point (Showman \& Guillot 2002, ....), which was subsequently confirmed by observation (Knutson et al. 2007). The wind whose speed of $1km \ s^{-1}$ is also detected by blueshift of CO, although the process is still preliminary (Snell et al. 2010). 

Analytical approach also catch up along the numerical study. Showman \& Polvani (2010, 2011) had laid a theoretical basis for the circulation of Hot Jupiter, which sufficiently explained the fast equatorial zonal jet. It turns out that this jet is because ... + ... and .... plus ..... This results is subsequently extended and confirmed by Tsai. et al (2014). The fragmented physics piece of hot jupiter atmosphere is being elucidated and assembled.

\begin{itemize}
\item With what group?
\item What are the charateristics of such simulations?
\end{itemize}

\item Several interesting features of hot jupiter has been demonstrated in the simulation. 
\begin{itemize}
\item Most notably is the fast zonal wind spanning from the equator up to 10 - 60 degree latitude with very high speed (3-4 km/s) 
\begin{itemize} 
\item Heat transport from day to night side
\item Superrotating. Analytical Explanation. 
\end{itemize}
\item What else
\end{itemize}

\item Prominent unanswered question?

Nonetheless, details of the picture when closely examined are still missing  (see Heng \& Showman 2014 for a comprehensive review). One of which is the inflated Hot Jupiter  problem. It has been observed that hot Jupiter appear to have larger radii when they are more irradiated (Baraffe, Chabrier \& Barman 2010). Most proposed explanation require an interior power source that would contribute to the radiated heat from gravitational contraction to reach a thermal equilibrium with a larger radius. 
Some proposed mechanism are the dynamical deposition of heat vertically inward via waves or advection (Guillot \& Showman 2002), tidal heating (Bodenheimer et al. 2001, 2003) and Ohmic dissipation, first clearly stated by Batygin \& Stevenson (2010).  


\begin{itemize}
\item Inflated radius problem
\begin{itemize}
\item Tidal heating (Bodenheimer et al. 2001, 2003)
\item Turbulence
\item Shock?
\item MHD
\end{itemize}
\item What else?
\end{itemize}

%5
\item MHD effect on the atmosphere

In strongly-irradiated atmosphere, the temperature ~ 1000 - 3000k in the upper atmosphere is not sufficient to ionize H or He significantly but high enough to paritally ionize alkali metals such as Na and K. If the planet possesses a magnetic field, it will resist the motion advected perpendicular to the field line, turn part of the kinectic energy into heat. Therefore magnetic field are speculated to leave at least 2 major inprints on the atmospheric circulation picture. First, the magnetic field could slow down the zonal wind, reducing the thermal transportation efficiency from the dayside to night side. Second, induced current generated by the magnetic field effect on the fast zonal flow is a heat source.  This source could be the missing heating mechanism behind the inflated radii. Whether it works out in detail remains a topic of debate. 

(..., .., ...)

This is a logical first step to address the problem


\begin{itemize}
\item MHD slows down the jet, slow down the heat transfer, maintain a high temperture difference in night and day side
\item Ohmic heating might be a heat source to inflate the radius
\item What have others done?
\end{itemize}



\item What is my work about?

To date, some works have been published to attack the problem. Perna et al. (2010a,b) adopt a kinematic approach by treating the Lorentz force as a rayleigh drag which is a velocity-dependent force opposing the flow. This Lorentz force is estimated from the induced current and a precribed dipole magnetic field. The Ohmic heating power could be evaluated from the dissipation by this drag force. Rauscher \& Menou (2013) improved this model by adding the self-consistent spatial variation of the electrical conductivity, thereby to the Lorentz force. The results show a significant decrease around ~ 30 \% in peak-velocity of the zonal flow, which could be important in hindering the formation efficiency of shock. However, by the nature of this treatment, the general struture of the flow is largely unchanged from the hydrodynamical model. Although this might be true physically, the reverse also remains possible - the case that the simplified Lorentz force in these dragged 3D GCMs could not capture the essence of the magnetic effect. 

Specifically, in these prescription, the feedback of the flow onto the magnetic field is not properly considered, which means that the induced field by the current is neglected. This approach is only valid when magnetic Reynolds number $\leq$ 1 ($R_m = UL/\eta$, where  $U$ is a typical speed and $L$ is typical length scale and $\eta$ is the magnetic diffusitivity). This is often not true in the Hot jupiter atmospheres.

A full self-consistently MHD treatment is indeed needed. Batygin, Stanley \& Setevenson (2013) and Rogers \& Showman (2014) set a step forward when they solve a set of MHD equations, however, with compromise. The former group utilizes Bussinesq approximation, which essentially render their model 2-dimensional and Ohmic heating is not included; Anelastic approximation and non-self-consistent magnetic resitivity are used in the latter paper. Both approximations cannot handle properly the supersonic flow. The purpose of this work is to overcome these challenges by solving a full compressible self-consistent MHD equations. 

For this purposed 

We use the finite volume scheme DUMSES that is based on the Godunov method (Toro 1997) 


\begin{itemize}
\item Honestly, I have no idea
\item To compare with others' present work. Are they reliable? $=>$ Point out their lacking
\begin{itemize}
\item Show the superiorty of Dumses:
\begin{itemize}
\item Fully compressible code
\end{itemize}
\item Have varying resitivity with pressure and compare 
\end{itemize}
\end{itemize}

\end{enumerate}


\end{document}