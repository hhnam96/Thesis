\documentclass[11pt]{article}

\begin{document}

\section{Introduction}

\begin{enumerate}

\item The discovery of Gas Giant Planet in recent year lay the framework and constraint for the analytical and numerical study. 

The field of exoplanet research has been expanding rapidly in the recent decade, enabled by the progress of observation in terms of sheer quantity and quality. Transiting and radial velocity emerge as the main workhorse, having been used i tandem to calculate the orbital and structural features of these exoplanets.  Complementarily, Transmission and emission spectroscopy provide the useful constraint of the temperature profile and atmospheric composition (Heng \& Showman 2015)  Even the observables related to the dynamics like the brightness distribution (Knutson et al. 2007) and wind speed (Snell et al. 2010) are also detected 

The planets are in close proximity with their star are refered to as Hot Jupiter.  Hot Jupiter are often tidal locked, This type of planet are of particular interest. Due to their close proximity to the star, they receive intensive stellar irradiation, which form a distinct radiative layer atop convective interiors and drives atmospheric wind to unusual high speed. Because of being tidal locked, their rotation period is equal to their orbital period, leading to a distinct and permanet day side and night side. Such asymmetric heating drives strong wind for heat redistrubution. 

\begin{itemize}
\item Brief characteristics of hot jupiter
\begin{itemize}
\item Tidal lock, so its rotational period is equal to its orbital period. Leading to a distinct day and night side of a planet.
\item Close to the central star => high stellar flux => hot, strong radiative forcing 
\item Roughly same mass with jupiter
\end{itemize}
\item With what mission?
\item With what technique?
\item What have been observed?
\end{itemize}

\item Numerical simulation of such planets have been done by several group with improvement is ongoing implemented.
All these  constraint lay the framework for numerical study. Thus far, atmospheric circulation has been model by several groups(Showman et al. 2008, ...). These global circulation models have produced some robust and prominent results such as the transonic wind speed and most notably is the broad eastward equatorial zonal jet near the photosphere. This zonal wind often flow with the velocity of order few kilometer per second, which is larger than speed of sound by a factor of a few. With this zonal wind, the brightest spot are predicted to be advected eastward of the substellar point (Showman \& Guillot 2002, ....), which was subsequently confirmed by observation (Knutson et al. 2007). The wind whose speed of $1km \ s^{-1}$ is also detected by blueshift of CO, although the process is still preliminary (Snell et al. 2010). 

\begin{itemize}
\item With what group?
\item What are the charateristics of such simulations?
\end{itemize}

\item Several interesting features of hot jupiter has been demonstrated in the simulation. 
\begin{itemize}
\item Most notably is the fast zonal wind spanning from the equator up to 10 - 60 degree latitude with very high speed (3-4 km/s) 
\begin{itemize} 
\item Heat transport from day to night side
\item Superrotating. Analytical Explanation. 
\end{itemize}
\item What else
\end{itemize}

\item Prominent unanswered question?
\begin{itemize}
\item Inflated radius problem
\begin{itemize}
\item Tidal heating (Bodenheimer et al. 2001, 2003)
\item Turbulence
\item Shock?
\item MHD
\end{itemize}
\item What else?
\end{itemize}

%5
\item MHD effect on the atmosphere
\begin{itemize}
\item MHD slows down the jet, slow down the heat transfer, maintain a high temperture difference in night and day side
\item Ohmic heating might be a heat source to inflate the radius
\end{itemize}


\item What is my work about?
\begin{itemize}
\item Honestly, I have no idea
\item To compare with others' present work. Are they reliable? $=>$ Point out their lacking
\begin{itemize}
\item Show the superiorty of Dumses:
\begin{itemize}
\item Fully compressible code
\end{itemize}
\item Have varying resitivity with pressure and compare 
\end{itemize}
\end{itemize}

\end{enumerate}


\end{document}