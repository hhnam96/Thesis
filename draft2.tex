\documentclass[11pt]{article}

\usepackage[margin=1in,includefoot]{geometry}

\begin{document}

\section{Introduction}

\begin{enumerate}

\item Observations 

The field of exoplanet research has been expanding rapidly in the recent decade, enabled by the progress of observation in terms of quantity and quality. Transiting and radial velocity emerge as the main workhorse, having been used in tandem to calculate the orbital and structural features of these exoplanets. In addition, transmission and emission spectroscopy provide a useful constraint of the temperature profile and atmospheric composition (Heng \& Showman 2015). Observables related to the dynamics like the brightness distribution (Knutson et al. 2007) and preliminary insight about wind speed in the atmospherere from the blueshift of CO line (Snellen et al. 2010) have also been found. On top of that, new instruments such as James Webb Space Telescope will soon provide more important constraints, making theoretical study of exoplanet atmospheric circulation more credible and attractive. 

\item Hot Jupiter

Among the exoplanets, hot Jupiters are of particular interest because of their distinct characteristics. They are gas giants located close to their star (within $\sim$ 0.1 AU - Where AU is Astronomical Unit denoting the Earth-Sun distance), enabling transit observation much easier due to geometry. Also, the secondary eclipse depth is more visible because of the hot atmosphere.  The transit methods measures the diminution of brightness when then planet passing in front of the star and secondary eclipse method measures the change when it passes behind.

The close distance to the central star makes the irradiation on hot Jupiter unusually high. This intensive stellar irradiation renders their atmosphere dynamic and vibrant. A distinct radiative layer is thus formed above the convective interior and drives atmospheric wind to unusual high speed. On top of that, being tidally locked - the rotation period is equal to orbital period, leading to permanent day side and night side. Such asymmetric heating drives strong wind for heat redistribution. 

Overall, the thermal and and dynamical responses to the unusual high thermal forcing make hot Jupiter atmosphere unique and intriguing for theoretical study.

\item Inflated hot Jupiter problem

Nonetheless, details of the picture when closely examined are still missing  (see Heng \& Showman 2014 for a comprehensive review). One of which is the inflated hot Jupiter  problem. It has been observed that hot Jupiters appear to have larger radii when they are more irradiated (Baraffe, Chabrier \& Barman 2010). Most proposed explanation require an interior power source that would contribute to the radiated heat from gravitational contraction to reach a thermal equilibrium with a larger radius. 
Some proposed mechanism are the dynamical deposition of heat vertically inward via waves or advection (Guillot \& Showman 2002), tidal heating (Bodenheimer et al. 2001, 2003) and Ohmic dissipation, first clearly stated by Batygin \& Stevenson (2010).  

\item Hydrodynamical models

To date, the atmospheric circulation of hot Jupiter has been modeled by several groups (Showman \& Guillot 2002, Showman et al. 2008, Dobbs-Dixon \& Lin 2008, Rauscher \& Menou 2010, Heng et al. 2011, Fromang et al. 2016). Some robust and consistent results were  prominent in these Global Circulation Models (GCM) such as the presence of transonic wind speed and most notably a broad eastward equatorial zonal jet near the photosphere. This zonal wind flows with the velocity of order few kilometer per second, which is larger than speed of sound by a factor of a few. With this zonal wind, the brightest spot was predicted to be advected eastward of the substellar point (Showman \& Guillot 2002) which was subsequently confirmed by observations (Knutson et al. 2007).  Wind whose speed of $1km \ s^{-1}$ has also been quantatively detected by blueshift of CO lines, although the process is still preliminary (Snellen et al. 2010). 

Analytical approaches also catch up with the numerical study. Following the work of Matsuno (1966) and Gill (1980), Showman \& Polvani (2010, 2011) had laid a theoretical basis for the atmospheric circulation of Hot Jupiters, which explained the fast equatorial zonal jet. The authors concluded that the thermally-forced standing planetary-scale Rossby wave and Kelvin waves are responsible for transporting angular momentum equatorward and up-gradient from the high-latitude region, which is in turn balanced by the drag and vertical momentum transport. This result is subsequently extended into a more complete three-dimentional theory by Tsai, Dobbs-Dixon and Gu (2014). The fragmented physics piece of hot jupiter atmosphere is being elucidated and assembled.

\item MHD effect on the atmosphere

In strongly-irradiated atmosphere, the temperature ~ 1000 - 3000 K in the upper atmosphere is not sufficient to ionize H or He significantly but high enough to paritally ionize alkali metals such as Na and K. If the planet possesses a magnetic field, it will resist the motion advected perpendicular to the field line, turning part of the kinectic energy into heat. Therefore magnetic field have been speculated to leave at least 2 major effects on the picture of atmospheric circulation. First, they could slow down the zonal wind, reducing the thermal transport efficiency from the day side to night side. Second, the magnetic field could dissipates the the kinetic energy to heat. This heat source has been suggested as the missing heating mechanism behind the inflated radii (Batygin \& Stevenson 2010). Whether it works out or not remains a topic of debate. 


\item Review of the literature

To date, some works have been published to attack the problem. Perna et al. (2010a,b) and Rauscher \& Menou (2013) model the effect of magnetic field as a rayleigh drag which is a velocity-dependent force opposing the flow. This drag is included in the pre-existing hydrodynamical model (Rauscher \& Menou 2010). The Ohmic heating could be thereby evaluated from the dissipation by this drag force. The results show a significant decrease  $\sim$ 30 \% of the peak velocity of the zonal flow, which could be important in hindering the formation efficiency of shock. This is a logical first step to address the problem. However, due to the nature of this treatment, the general struture of the flow is largely unchanged compared to the purely hydrodynamical model. Although this might be true physically, the reverse also remains possible - the case that the simplified Lorentz force in these dragged 3D GCMs may not be adaquate to capture the essence of the magnetic effect. 

Specifically, in these prescription, the feedback of the flow onto the magnetic field is not properly considered. This approach is only valid for magnetic Reynolds number $\leq$ 1 ($R_m = UL/\eta$, where  $U$ is a typical speed and $L$ is typical length scale and $\eta$ is the magnetic diffusitivity). This is often not true in the Hot jupiter atmospheres.

A full self-consistently MHD treatment is indeed needed. Batygin, Stanley \& Setevenson (2013) and Rogers \& Showman (2014) make a step forward by solving a set of MHD equations. However, they also compromise. The former group utilizes Bussinesq approximation, which essentially renders their model 2-dimensional and Ohmic heating is not included; 

Anelastic approximation and non-self-consistent magnetic resitivity are used in the latter paper. Both approximations cannot handle properly the supersonic flow. The purpose of this work is to overcome these challenges by solving a full compressible self-consistent MHD equations. 

\end{enumerate}

\section{Models and Numerical Implementation }
\begin{enumerate}
\item The induction equation + lorentz force 
\item The governing equations
\item The buffer region 

Why needs it? Justify its validity
\item Numerical Implementation 
\end{enumerate}

\section{Results}
\begin{enumerate}
\item Low resolution model
\begin{enumerate}
\item Varrying magnetic resistivity
\item With different buffer
\item The forces
\end{enumerate}
\item High resolution
\end{enumerate}

\section{Discussion}
\begin{enumerate}
\item Compares with others'results
\item The west ward jet 
\item Weak point of this model
\item Improvement?
\end{enumerate}


\end{document}